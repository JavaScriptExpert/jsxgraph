% Reference Card for JSXGraph
% To create a document out of this for printing, save this text into
% a file 'jsxgraph_refcard.tex' and run the commands
%     tex jsxgraph_refcard.tex
%     dvips -t landscape jsxgraph_refcard.dvi -o
%     gs -sDEVICE=pdfwrite -dBATCH -dNOPAUSE -sOutputFile=jsxgraph_refcard.pdf jsxgraph_refcard.ps
% this will result in a PDF file 'jsxgraph_refcard.pdf'
% this is a comment

%**start of header
\newcount\columnsperpage
\overfullrule=0pt

% This file can be printed with 1, 2, or 3 columns per page (see below).
% [For 2 or 3 columns, you'll need 6 and 8 point fonts.]
% Specify how many you want here.  Nothing else needs to be changed.

\columnsperpage=3

% This reference card is distributed in the hope that it will be useful,
% but WITHOUT ANY WARRANTY; without even the implied warranty of
% MERCHANTABILITY or FITNESS FOR A PARTICULAR PURPOSE.  

% This file is intended to be processed by plain TeX (TeX82).
%
% The final reference card has six columns, three on each side.
% This file can be used to produce it in any of three ways:
% 1 column per page
%    produces six separate pages, each of which needs to be reduced to 80%.
%    This gives the best resolution.
% 2 columns per page
%    produces three already-reduced pages.
%    You will still need to cut and paste.
% 3 columns per page
%    produces two pages which must be printed sideways to make a
%    ready-to-use 8.5 x 11 inch reference card.
%    For this you need a dvi device driver that can print landscape
% Which mode to use is controlled by setting \columnsperpage above.

%
% Thanks:
%  (reference card macros due to Stephen Gildea)

\def\versionnumber{0.1}  % Version of this reference card
\def\year{2009}
\def\month{Feb}
\def\version{\month\ \year\ v\versionnumber}

\def\shortcopyrightnotice{\vskip .5ex plus 2 fill
  \centerline{\small \copyright\ \year\ v\versionnumber}}

\def\copyrightnotice{\vskip 1ex plus 100 fill\begingroup\small
\centerline{\version.}
\endgroup}

% make \bye not \outer so that the \def\bye in the \else clause below
% can be scanned without complaint.
\def\bye{\par\vfill\supereject\end}

\newdimen\intercolumnskip
\newbox\columna
\newbox\columnb

\def\ncolumns{\the\columnsperpage}

\message{[\ncolumns\space 
  column\if 1\ncolumns\else s\fi\space per page]}

\def\scaledmag#1{ scaled \magstep #1}

% This multi-way format was designed by Stephen Gildea
% October 1986.
\if 1\ncolumns
  \hsize 4in
  \vsize 10in
  \voffset -.7in
  \font\titlefont=\fontname\tenbf \scaledmag3
  \font\headingfont=\fontname\tenbf \scaledmag2
        \font\headingfonttt=\fontname\tentt \scaledmag2
  \font\smallfont=\fontname\sevenrm
  \font\smallsy=\fontname\sevensy

  \footline{\hss\folio\hss}
  \def\makefootline{\baselineskip10pt\hsize4in\line{\the\footline}}
\else
  \hsize 3.2in
  \vsize 7.5in %% was 7.95in
  \advance\vsize by 1cm
  \hoffset -12mm %-2mm % was -.75in
  \voffset -.745in
  \font\titlefont=cmbx10 \scaledmag2
  \font\headingfont=cmbx10 \scaledmag1
  \font\headingfonttt=cmtt10 \scaledmag1
  \font\smallfont=cmr6
  \font\smallsy=cmsy6
  \font\eightrm=cmr8
  \font\eightbf=cmbx8
  \font\eightit=cmti8
  \font\eighttt=cmtt8
  \font\eightsy=cmsy8
  \font\eightsl=cmsl8
  \font\eighti=cmmi8
  \font\eightex=cmex10 at 8pt
  \textfont0=\eightrm  
  \textfont1=\eighti
  \textfont2=\eightsy
  \textfont3=\eightex
  \def\rm{\fam0 \eightrm}
  \def\bf{\eightbf}
  \def\it{\eightit}
  \def\tt{\eighttt}
  \def\sl{\eightsl}
  \normalbaselineskip=.8\normalbaselineskip
  \normallineskip=.8\normallineskip
  \normallineskiplimit=.8\normallineskiplimit
  \normalbaselines\rm       %make definitions take effect

  \if 2\ncolumns
    \let\maxcolumn=b
    \footline{\hss\rm\folio\hss}
    \def\makefootline{\vskip 2in \hsize=6.86in\line{\the\footline}}
  \else \if 3\ncolumns
    \let\maxcolumn=c
    \nopagenumbers
  \else
    \errhelp{You must set \columnsperpage equal to 1, 2, or 3.}
    \errmessage{Illegal number of columns per page}
  \fi\fi

  \intercolumnskip=.46in
  \def\abc{a}
  \output={%
      % This next line is useful when designing the layout.
      %\immediate\write16{Column \folio\abc\space starts with \firstmark}
      \if \maxcolumn\abc \multicolumnformat \global\def\abc{a}
      \else\if a\abc
    \global\setbox\columna\columnbox \global\def\abc{b}
        %% in case we never use \columnb (two-column mode)
        \global\setbox\columnb\hbox to -\intercolumnskip{}
      \else
    \global\setbox\columnb\columnbox \global\def\abc{c}\fi\fi}
  \def\multicolumnformat{\shipout\vbox{\makeheadline
      \hbox{\box\columna\hskip\intercolumnskip
        \box\columnb\hskip\intercolumnskip\columnbox}
      \makefootline}\advancepageno}
  \def\columnbox{\leftline{\pagebody}}

  \def\bye{\par\vfill\supereject
    \if a\abc \else\null\vfill\eject\fi
    \if a\abc \else\null\vfill\eject\fi
    \end}  
\fi

% we won't be using math mode much, so redefine some of the characters
% we might want to talk about
%\catcode`\^=12
%\catcode`\_=12
\catcode`\~=12

\chardef\\=`\\
\chardef\{=`\{
\chardef\}=`\}
\chardef\underscore=`\_
\chardef\'="0D % These are upright quote marks


\hyphenation{}

\parindent 0pt
\parskip .85ex plus .35ex minus .5ex

\def\small{\smallfont\textfont2=\smallsy\baselineskip=.8\baselineskip}

\outer\def\newcolumn{\vfill\eject}

\outer\def\title#1{{\titlefont\centerline{#1}}\vskip 1ex plus .5ex}

\outer\def\section#1{\par\filbreak
  \vskip .5ex  minus .1ex {\headingfont #1}\mark{#1}%
  \vskip .3ex  minus .1ex}

\outer\def\librarysection#1#2{\par\filbreak
  \vskip .5ex  minus .1ex {\headingfont #1}\quad{\headingfonttt<#2>}\mark{#1}%
  \vskip .3ex  minus .1ex}


\newdimen\keyindent

\def\beginindentedkeys{\keyindent=1em}
\def\endindentedkeys{\keyindent=0em}
\def\begindoubleindentedkeys{\keyindent=2em}
\def\enddoubleindentedkeys{\keyindent=1em}
\endindentedkeys

\def\paralign{\vskip\parskip\halign}

\def\<#1>{$\langle${\rm #1}$\rangle$}

\def\kbd#1{{\tt#1}\null}    %\null so not an abbrev even if period follows

\def\beginexample{\par\vskip1\jot
\hrule width.5\hsize
\vskip1\jot
\begingroup\parindent=2em
  \obeylines\obeyspaces\parskip0pt\tt}
{\obeyspaces\global\let =\ }
\def\endexample{\endgroup}

\def\Example{\qquad{\sl Example\/}.\enspace\ignorespaces}

\def\key#1#2{\leavevmode\hbox to \hsize{\vtop
  {\hsize=.75\hsize\rightskip=1em
  \hskip\keyindent\relax#1}\kbd{#2}\hfil}}

\newbox\metaxbox
\setbox\metaxbox\hbox{\kbd{M-x }}
\newdimen\metaxwidth
\metaxwidth=\wd\metaxbox

\def\metax#1#2{\leavevmode\hbox to \hsize{\hbox to .75\hsize
  {\hskip\keyindent\relax#1\hfil}%
  \hskip -\metaxwidth minus 1fil
  \kbd{#2}\hfil}}

\def\threecol#1#2#3{\hskip\keyindent\relax#1\hfil&\kbd{#2}\quad
  &\kbd{#3}\quad\cr}


%**end of header

\title{JSXGraph Reference Card}

\section{Include JSXGraph in HTML}
Three parts are needed: Include files containing the software, an HTML element, and 
JavaScript code.

{\bf Include files:}

Three files have to be included:
{\tt jsxgraph.css}, {\tt jsxgraphcore.js} and either {\tt prototype.js} or {\tt jquery.js}.

\metax{$<$link rel="stylesheet" type="text/css" href="xxx/jsxgraph.css"/$>$}{}
\metax{$<$script type="text/javascript" src="xxx/prototype.js"$>$$<$/script$>$}{}
\metax{$<$script type="text/javascript" src="xxx/jsxgraphcore.js"$>$$<$/script$>$}{}

or

\metax{$<$link rel="stylesheet" type="text/css" href="xxx/jsxgraph.css"/$>$}{}
\metax{$<$script type="text/javascript" src="xxx/jquery.min.js"$>$$<$/script$>$}{}
\metax{$<$script type="text/javascript" src="xxx/jsxgraphcore.js"$>$$<$/script$>$}{}

{\tt xxx} is the location of the files. This can be a local directory or
{\tt http://jsxgraph.uni-bayreuth.de/distrib/}

{\bf HTML element containing the construction:}

\metax{$<$div id="box" class="jxgbox"}{}
\metax{\phantom{xxx}style="width:600px; height:600px;"$>$$<$/div$>$}{}

{\bf JavaScript code:}

\metax{$<$script type="text/javascript"$>$}{}
\metax{\phantom{xxx}{\tt var brd = JXG.JSXGraph.initBoard('box',$\{$axis:true$\}$);}}{}
\metax{$<$script$>$}{}

\section{Initialize the board}
\metax{}{var brd = JXG.JSXGraph.initBoard('box',$\{$attributes$\}$);}

\section{Basic commands}
\metax{{\tt var el = brd.createElement('type',[parents],$\{$attributes$\}$);}}{}
\metax{{\tt el.setProperty($\{$attributes$\}$);}}{}

\section{Available Elements}
'angle', 'arc', 'arrow', 'arrowparallel', 'axis',
'bisector',
'chart', 'circumcircle', 'circumcirclemidpoint',
'curve', 'circle', 'glider', 'group', 'image', 'integral',
'line', 'midpoint', 'mirrorpoint',
'normal', 'parallel', 'parallelpoint', 'perpendicular', 'perpendicularpoint', 'polygon', 
'point', 'reflection', 'sector', 'slider', 'spline', 'tangent', 'text', 'ticks', 'transform', 'turtle'

\section{Point}
\metax{{\tt brd.createElement('point',[parents],$\{$attributes$\}$);}}{}

{\bf Parent elements:}

\metax{Euclidean coordinates}{[3,-2]}
\metax{Homogeneous coords ($z$ in first place)}{[1, 3,-2]}
\metax{Functions for $x,y$, (and $z$)}{[function()$\{$return p1.X();$\}$,}
\metax{}{function()$\{$return p2.Y();$\}$]}

{\bf Methods}

\metax{$x$-coordinate}{p.X()}
\metax{$y$-coordinate}{p.Y()}
\metax{(Homogeneous) $z$-coordinate}{p.Z()}
\metax{Distance to other point}{p.Dist(q)}

\section{Glider}
\metax{{\tt brd.createElement('glider',[parents],$\{$attributes$\}$);}}{}

{\bf Parent elements:}

\metax{Initial coordinates and object to glide on}{[3, -2, c]}
\metax{Object to glide on (initially at origin)}{[c]}

Coordinates may also be defined by functions, see Point.


\section{Line}
\metax{{\tt brd.createElement('line',[parents],$\{$attributes$\}$);}}{}

{\bf Parent elements:}

\metax{2 points}{[p1,p2]}
\metax{3 coordinates (can also be functions)}{[c,a,b]}

In case of coordinates as parents, the line is the set of solutions
of 
$$ a\cdot x+b\cdot y+c\cdot z=0.$$

\section{Curve}
The supported curve types are: 

\metax{Function graph}{}
\metax{{\tt brd.createElement('functiongraph',[parents],$\{$attributes$\}$);}}{}

\metax{Parameter curve}{'parameter'}
\metax{Data plot}{'plot'}
\metax{Polar curve}{'polar'}
\metax{{\tt brd.createElement('curve',[parents],$\{$attributes$\}$);}}{}

{\bf Parent elements:}

{\sl -- Function graph:}\par
\metax{function [, start, end]}{[function(x)$\{$return x*x;$\}$,-1,1]}

{\sl -- Parameter curve:}\par
\metax{$x$ function, $y$ function [, start, end]}{}
\metax{}{[function(x)$\{$return x;$\}$,function(x)$\{$return x*x;$\}$]}

{\sl -- Data plot:}\par
\metax{array of $x$-coordinates,}{}
\metax{\phantom{xxx}array of $y$-coordinates,}{[[1,2,3],[4,-2,3]]}
\metax{or array of $x$-coordinates, function}{}
\metax{}{[[1,2,3],function(x)$\{$return x*x;$\}$]}

{\sl -- Polar curve:}\par
Defined by the equation $r=f(\phi)$.

\metax{Defining function, [offset, start, end]}{[f,[1,2],0,Math.PI]}

\section{Circle}
\metax{{\tt brd.createElement('circle',[parents],$\{$attributes$\}$);}}{}

{\bf Parent elements:}

\metax{2 points}{[p1, p2]}
\metax{point,radius (constant or function)}{[p, r]}
\metax{point,circle}{[p, c]}
\metax{circle,point}{[c, p]}
\metax{point,line segment}{[p, l]}
\metax{line segment, point}{[l, p]}

\section{Turtle}
\metax{{\tt brd.createElement('turtle',[],$\{$attributes$\}$);}}{}
\metax{{\tt var t = brd.createElement('turtle',[parents],$\{$attributes$\}$);}}{}

The turtle has a position and a direction (in degrees). All angles have
to be supplied in degrees.

{\bf Parent elements:}

\metax{Optional start values for $x$, $y$, direction}{[1,1,70]}

{\bf Methods:}

Most of the methods have an abbreviated alternative version.

\metax{{\tt t.forward(len); t.fd(len);}}{}
\metax{{\tt t.back(len); or t.bk(len);}}{}
\metax{{\tt t.right(angle); or t.rt(angle);}}{}
\metax{{\tt t.left(angle); or t.lt(angle);}}{}
\metax{{\tt t.penUp(); or t.pu();}}{}
\metax{{\tt t.penDown(); or t.pd();}}{}
\metax{{\tt t.clearScreen(); or t.cs();}}{}
\metax{{\tt t.clean();}}{}
\metax{{\tt t.setPos(x,y); }}{}
\metax{{\tt t.home();}}{}
\metax{{\tt t.hideTurtle(); or t.ht();}}{}
\metax{{\tt t.showTurtle(); or t.st();}}{}
\metax{{\tt t.setPenSize(size);}}{}
\metax{{\tt t.setPenColor(col);}}{} (col: colorString, e.g. 'red' or '\#ff0000')

\metax{{\tt t.setProperty({key1:value1,key2:value2,...});}}{}
\metax{{\tt t.pushTurtle();}}{}
\metax{{\tt t.popTurtle();}}{}
\metax{{\tt t.lookTo([x,y]);}}{} (Turtle looks to a coordinate pair. If t2 is another turtle object: t.lookTo(t2.pos))

\metax{{\tt t.lookTo(dir);}}{} (Turtle looks into a given direction)

\metax{{\tt t.moveTo([x,y]);}}{} 


\section{Attributes of geometric elements}


\section{Links}
Help pages are available at
{\tt http://jsxgraph.org}


%%%%%%%%%%%%%%%%%%%%%%%%%% END LIBRARIES %%%%%%%%%%%%%%%%%%%%%%%%%%%%%%%%%%

% This goes at the bottom of the last page (column 6)
\copyrightnotice
%

\bye

% Local variables:
% compile-command: "tex wiki-refcard.tex"
% End:
