\documentclass[10pt]{article}
\usepackage{a4wide}
\usepackage[latin1]{inputenc}
\usepackage[ngerman]{babel}
\usepackage[ngerman]{varioref}
\usepackage[T1] {fontenc}
\usepackage[latin1] {inputenc}
\usepackage{amsmath}
\usepackage{amssymb}
\usepackage{verbatim}
\usepackage{pst-all}
\usepackage{float}
%\usepackage{pst-pdf}
\usepackage{multicol}
\usepackage{multirow}
\usepackage{tabularx}


\usepackage{geometry}
\geometry{a4paper,left=30mm,right=30mm, top=2cm, bottom=2cm}

\newcommand{\R}{\mathbb{R}}
\newcommand{\Q}{\mathbb{Q}}
\newcommand{\Z}{\mathbb{Z}}
\newcommand{\N}{\mathbb{N}}
\newcommand{\C}{\mathbb{C}}
\newcommand{\PC}{\mathcal{P}}
\newcommand{\ZC}{\mathcal{Z}}
\newcommand{\Sum}{\sum\limits}
\newcommand{\Prod}{\prod\limits}
\newcommand{\Lim}{\lim\limits}
\newcommand{\Int}{\int\limits}
\renewcommand{\Im}{\text{Im}}
\renewcommand{\Re}{\text{Re}}
\newcommand{\eps}{\varepsilon}
\parindent 0pt
\pagestyle{empty}


\newcommand{\luecke}[1]{\raisebox{-6pt}{\makebox[#1]{\dotfill}}}

\def\versionnumber{0.81}  % Version of this reference card
\def\year{2010}
\def\month{Juni}
\def\version{\month\ \year\ v\versionnumber}

\begin{document}
\title{JessieScript Reference (Version \version)}
\begin{center} {\LARGE\textbf {JessieScript Reference (Version
\version)}}
\end{center}
\section{Construct}
Easy mathematical constructions can be created with the line
\begin{verbatim}
     board.construct(...);
\end{verbatim} Within the brackets its possible to pass different
elements, seperated by semicolon, as one string. Blanks are
irrelevant.
\\\\ Possible elements are: \\ \begin{tabular}{|l|l|} \hline \\[-0.75em] {\large
\textbf{Example}} & {\large \textbf{Description}} \\
\hline\hline \verb+A(1,1)+ & point at (1,1) with name \verb'A'
\\ \hline \verb+BB(-2|0.5)+ & point at
(-2,0.5) with name \verb'BB' \\
\hline\verb+]AB[+ & straight line through points \verb'A' and \verb'B' \\
\hline\verb+[AB[+ &
ray through points \verb'A' and \verb'B', stopping at \verb'A' \\
\hline\verb+]A BB]+ & ray through points \verb'A' and \verb'BB',
stopping at \verb'BB'
\\ \hline\verb+[AB]+ & segment between \verb'A' and \verb'B' \\
\hline \verb+g=[AB]+ & segment between \verb'A' and \verb'B' with name \verb'g' \\
\hline\verb+k(A,4)+ & circle with midpoint \verb'A' and radius 4 \\
\hline\verb+k(A,[BC])+ & circle with midpoint \verb'A', whose
radius is given by the (not necessarily
\\ & existing) segment \verb'[BC]' \\ \hline\verb+k(A,B)+ & circle with midpoint \verb'A',
through point \verb'B' \\
\hline\verb+k1=k(A,3)+ & circle with midpoint \verb'A' with radius
3 with name \verb'k1'
\\ \hline\verb+P(g)+ & glider \verb'P' on the object \verb'g' \\
\hline\verb+Q(k1,0,1)+ & glider \verb'Q' on the object \verb'k1'
at (0,1) \\ \hline \verb+g&k1+ & intersection point(s) of the
objects \verb'g' and \verb'k1' \\ \hline \verb+S=g&k1+ &
intersection point(s) of the \verb'g' and \verb'k1'.  \\ &
Multiple intersection points
are named with \verb'S'$_1$ and \verb'S'$_2$, single ones with \verb'S'. \\
\hline\verb+||(A,g)+ & parallel line to \verb'g' through point
\verb'A'
\\  \hline\verb+|_(A,g)+ & perpendicular line to \verb'g' through point
\verb'A' \\ \hline\verb+<(A,B,C)+ & angle, defined by the points \verb'A', \verb'B', \verb'C' \\
\hline\verb+alpha=<(A,B,C)+ & angle, defined by the points
\verb'A', \verb'B', \verb'C', with name $\alpha$
\\  & Possible greek denominators are \verb'alpha',
\verb'beta', \verb'gamma', \verb'delta', \verb'epsilon', \\ &
\verb'zeta', \verb'eta', \verb'theta', \verb'iota', \verb'kappa',
\verb'lambda', \verb'mu', \verb'nu', \verb'xi', \verb'omicron',
\verb'pi', \verb'rho', \verb'sigmaf', \\ & \verb'sigma',
\verb'tau', \verb'upsilon', \verb'phi', \verb'chi', \verb'psi' and
\verb'omega'. \\ \hline\verb+1/2(A,B)+ & midpoint between \verb'A'
and \verb'B'
\\ \hline\verb+3/4(A,B)+ & point dividing the segment from \verb'A'
to \verb'B' at ratio 3:7 , \\ & i.e. $\frac{3}{4}$ parts of the
segment \verb'[AB]' are between \verb'A' and the constructed point
\\ & Therefore, any ratio of natural numbers is possible. \\
\hline\verb+P[A,B,C,D]+ & polygon through points \verb'A',
\verb'B', \verb'C', \verb'D' with name 'P' \\
\hline\verb'f:x^2+2*x' & functiongraph, $f:x\mapsto x^2+2\cdot x$ \\
\hline\verb'f:sin(x)' & functiongraph, $g:x\mapsto \sin(x)$ \\
\hline\verb'#Hello world(0,3)' & text \verb'Hello world' at (0,3)
\\ \hline
\end{tabular} \vspace*{0.5cm} \\
Its possible for every element (except points, graphs and
polygons) to provide a name directly by using
\begin{verbatim}
    objname = ...
\end{verbatim} \\ The function returns an object with all constructed elements, so that afterwards
the properties can still be changed.

\section{Fast modification of properties}
For setting the three most important properties there is a fast
possibility, all others have to be set afterwards by accessing the
particular elements and calling the corresponding function.
\\ These are \\
\begin{tabular}{|l|l|} \hline \\[-0.75em] {\large
\textbf{Property}} & {\large \textbf{Description}} \\
\hline\hline\verb+invisible+ & the object is invisible \\
\hline\verb+draft+ & the object is drawn in draft mode \\
\hline\verb+nolabel+ & the object does not have a label
\\ \hline
\end{tabular} \vspace*{0.5cm} \\
These properties are set directly at declaring the objects by
writing the respective key word (resp. key words, a combination is
possible), seperated by a blank behind the construction command
before the semicolon, i.e.
\begin{verbatim}
     P(1,1) nolabel; Q(2,3) draft nolabel; [PQ] invisible;
\end{verbatim}

\section{Zugriff auf Elemente}
Access to the elements after constructing them is possible by using: \\
\begin{tabular}
{|l|l|} \hline \\[-0.75em] {\large
\textbf{element}} & {\large \textbf{description}} \\ \hline\hline
\verb+constr.points[i]+ & take the $i$-th point or glider of the
construction \verb+constr+, \\ & also midpoints and dividing points are within this array \\
\hline\verb+constr.lines[i]+ & take the $i$-th line, ray or
segment of the construction \verb+constr+, \\ & also parallel and
perpendicular lines are within this array
\\
\hline\verb+constr.circles[i]+ & take the $i$-th cicle of the
construction \verb+constr+
\\
\hline\verb+constr.intersections[i]+ & take the $i$-th
intersection point of the construction \verb+constr+
\\
\hline\verb+constr.angles[i]+ & liefert take the $i$-th angle of
the construction \verb+constr+
\\
\hline\verb+constr.functions[i]+ & take the $i$-th function graph
of the construction \verb+constr+
\\
\hline\verb+constr.texts[i]+ & take the $i$-th text element of the
construction \verb+constr+
\\
\hline\verb+constr.polygons[i]+ & take the $i$-th polygon of the
construction \verb+constr+
\\
\hline\verb+constr.A+ & take the element with name \verb'A' of the
construction \verb+constr+
\\ \hline
\end{tabular}

\section{Macros}
Additionally it is possible to define macros. The key word is
\verb+Marco+, the parameters are, seperated by comma, provided
within round brackets, the content between curly braces. Left of
the equal sign any name can be given to the macro. \\ So the
syntax is given by
\begin{verbatim}
     macroName = Macro(param1, param2, param3, ...) { command1; command2; command3; ... };
\end{verbatim}
After that, the macro can be called by
\begin{verbatim}
     result = macroName(x1,x2,x3,...);
\end{verbatim}

\section{Example}
An example shall demonstrate the practical implementation.
\begin{verbatim}
    board = JXG.JSXGraph.initBoard('box', {originX: 50, originY: 300, unitX: 50,
                                           unitY: 50, axis:true});

    cons1 = board.construct("A(1,1);BC(1,3);k(A,[BC]);X(2,4)");
    cons2 = board.construct("J(7,4);l_2=[BC A]");

    cons1.points[0].face{'>'}; // A
    cons1.BC.strokeColor('black');
    cons2.l_2.strokeWidth(4);
    cons1.X.size(8);

    cons3 = board.construct("test = Macro(D,E,F) { g=[DE] nolabel; k1=k(D,[EF]);};
                                 ttt=test(A,X,J);");
    cons3.ttt.g.strokeColor('red');
\end{verbatim}

\end{document}
